
\section{Type-safety (CSTs only)}

\question[5] A major trend in programming is the use of powerful type systems to quickly rule out many runtime errors during compilation. In a simple type system, this may just involve checking that \emph{e.g.} a function which expects an integer as argument is not given a string. However, more advanced type systems have emerged. In DNH\footnote{You may be able to gain access to a language as powerful as DNH by using Haskell and enabling the \texttt{GADTs} and \texttt{EmptyDataDecls} extensions}, it is possible to define data types with no constructors (an empty data type):
\begin{displaymath}
\begin{array}{l}
\mathbf{datatype}~\mathit{foo}
\end{array}
\end{displaymath}
This declares a type named $\mathit{foo}$ which is equivalent to the empty set. In other words, there are no values of this type. DNH also allows programmers to specify the types of data constructors directly. Suppose you are given the following definition in ML:
\begin{displaymath}
\begin{array}{l}
\mathbf{datatype}~\alpha~\mathit{list} = \mathit{EMPTY} \mid \mathit{CONS}~\mathbf{of}~\alpha \times \alpha~\mathit{list}
\end{array}
\end{displaymath}
This definition introduces a new type called $\mathit{list}$ with one type parameter, and two constructors. Constructors are actually just values and functions in disguise so that they have types. For example, $\mathit{EMPTY}$ has type $\alpha~\mathit{list}$. In other words, $\mathit{EMPTY}$ is a value of the $\mathit{list}$ type. $\mathit{CONS}$, on the other hand, has type $\alpha \times \alpha~\mathit{list} \to \alpha~\mathit{list}$. In other words, it is a function which, given a value of type $\alpha$ and a list of elements of type $\alpha$, produces a list of elements of type $\alpha$. We observe that the type of a constructor is always just a function from the type of its parameters (if any) to the the type of the data type it is part of. In DNH, we can specify these types manually using the following syntax:
\begin{displaymath}
\begin{array}{l}
\mathbf{datatype}~\alpha~\mathit{list} = \mathit{EMPTY} : \alpha~\mathit{list} \mid \mathit{CONS} : \alpha \times \alpha~\mathit{list} \to \alpha~\mathit{list}
\end{array}
\end{displaymath}
On its own, this feature is not particularly useful, but it is if used with empty data types. Additionally, $\beta$ is a type parameter of $\mathit{bar}$ in the following example, but it is not used by the only constructor named $\mathit{BAZ}$. 
\begin{displaymath}
\mathbf{datatype}~\alpha~\beta~\mathit{bar} = \mathit{BAZ} : \alpha \to \alpha~\mathit{foo}~\mathit{bar} 
\end{displaymath}
One problem with functional lists is that we may, for example, try to take the head of an empty list, resulting in a runtime error. Using the three things you have learned about DNH above, define a new list type and associated functions ($\mathit{head}$, ...) so that it becomes impossible to call \emph{e.g.} $\mathit{head}$ on an empty list. \droppoints 
\begin{parts}
\part[2] Define two empty data types in DNH, one for each of the two possible states a list can be in. \droppoints 
\part[3] Complete the following definition. You should make use of the types from the previous question to annotate the $\mathit{list}$ type:
\begin{displaymath}
\begin{array}{lcl}
\mathbf{datatype}~\alpha~\beta~\mathit{list} & = & \mathit{EMPTY} :~ ??? \\
                                             & \mid & \mathit{CONS} :~ ???
\end{array}
\end{displaymath} \droppoints
\part[2] Define a type-safe $\mathit{head}$ function which can only be applied to non-empty lists. \droppoints 
\part[4] Discuss the advantages and disadvantages of this implementation of linked lists. Ideally, give an example of a function which can be defined for the usual linked lists, which cannot be defined for this implementation. \droppoints
\end{parts}
\section{Hutton's Razor strikes back}

\question \emph{Type inference} describes the process of assigning types to previously untyped terms of a formal language. A type may be seen as an approximation of the value of a term. We use types to try and catch errors in a program before it is run. For example, recall the definition of Hutton's razor:
\begin{displaymath}
e = e + e \mid n \in \mathbb{N}
\end{displaymath}
We extend this language with boolean values and conditionals:
\begin{displaymath}
e = e + e \mid n \in \mathbb{N} \mid b \in \mathbb{B} \mid \mathbf{if}~e~\mathbf{then}~e~\mathbf{else}~e
\end{displaymath}
Now let us define the evaluation function for this language:
\begin{displaymath}
\begin{array}{lcl}
\denote{\cdot} & : & e \to (\mathbb{N} + \mathbb{B})_{\bot}\\
\denote{n}         & = & \mathbf{inl}~n \\
\denote{b}         & = & \mathbf{inr}~b \\
\denote{e_0 + e_1} & = & \left\{ \begin{array}{ll}
\mathbf{inl}~(x + y) & \text{if } ~\denote{e_0}~ = \mathbf{inl}~x \text{ and } ~\denote{e_1}~ = \mathbf{inl}~y \\
\bot & \text{otherwise}
\end{array}  \right.\\
\denote{\mathbf{if}~e_0~\mathbf{then}~e_1~\mathbf{else}~e_2} & = & \left\{\begin{array}{ll}
\denote{e_1} & \qquad \quad \; \text{if } ~\denote{e_0}~ = \mathbf{inr}~\mathit{true}\\
\denote{e_2} & \qquad \quad \; \text{if } ~\denote{e_0}~ = \mathbf{inr}~\mathit{false}\\
\bot         & \qquad \quad \; \text{otherwise}
\end{array}\right.
\end{array}
\end{displaymath}
Note that the codomain of $~\denote{\cdot}~$ is $(\mathbb{N} + \mathbb{B})_{\bot}$. $\mathbb{N} + \mathbb{B}$ represents a \emph{tagged union}. Given two sets, $A$ and $B$, the tagged union of the two is defined as:
\begin{displaymath}
A + B = \{~\mathbf{inl}~a \mid a \in A~\} \cup \{~\mathbf{inr}~b \mid b \in B~\}
\end{displaymath}  
In other words, a tagged union is simply a union where the elements are tagged with either $\mathbf{inl}$ or $\mathbf{inr}$ to indicate which set they originate from.

The subscript $\bot$ indicates that the set is ``lifted''. This simply means that we include $\bot$ (read as \emph{bottom} or \emph{undefined}) in the set. In other words:
\begin{displaymath}
(\mathbb{N} + \mathbb{B})_{\bot} = \set{\bot, \mathbf{inl}~0,\mathbf{inl}~1, \ldots, \mathbf{inr}~\mathit{true}, \mathbf{inr}~\mathit{false}}
\end{displaymath}
To summarise, the co-domain of the evaluation function is the set of $\bot$, all natural numbers tagged with $\mathbf{inl}$, and all boolean values tagged with $\mathbf{inr}$. We use $\bot$ in the definition of the evaluation function to indicate failure: \emph{i.e.} things we cannot evaluate.

However, according to the grammar for $e$, a perfectly valid expression in this language would be the following:
\begin{displaymath}
\mathbf{if}~4~\mathbf{then}~\mathit{false}~\mathbf{else}~8
\end{displaymath}
This expression does intuitively not make much sense. If we apply the evaluation function to it, it will evaluate to $\bot$. To avoid such cases, we will add a type system to our language. We first need to define suitable types. A good start would be one type for each kind of value in our language: 
\begin{displaymath}
\tau = \mathbf{nat} \mid \mathbf{bool}
\end{displaymath}
We know that a well-formed expression will either evaluate to a natural number or to a boolean value. Since types are approximations of the values of expressions, our definition of $\tau$ seems adequate.

We now wish to define a binary relation $\vdash e : \tau$ which relates expressions $e$ to types $\tau$. For example, we want to say that all numbers in our language are of type $\mathbf{nat}$: $\vdash n : \mathbf{nat}$. Enumerating all of these relations would be impossible, however, so we declare rules according to which types are assigned different forms of expressions instead. We call this the \emph{deductive system}. Each rule in this system is presented in the following way:
\begin{mathpar}
\inferrule*[Right=RULE-NAME]
{ \mathit{condition1} \\ \mathit{condition2} \\ \ldots}
{ \mathit{conclusion}}
\end{mathpar}
These rules are like implications in logic: if all the conditions are true (shown above the bar), then we can draw the conclusion which is shown below the bar. The full deductive system for our language is now shown below:
\begin{mathpar}
\inferrule*[Right=T-VAL]
{ }
{
 \vdash n : \mathbf{nat}
} \and
\inferrule*[Right=T-BOOL]
{ }
{
 \vdash b : \mathbf{bool}
} \\
\inferrule*[Right=T-ADD]
{
	\vdash e_0 : \mathbf{nat} \\ \vdash e_1 : \mathbf{nat}
}
{
	\vdash e_0 + e_1 : \mathbf{nat}
} \and
\inferrule*[Right=T-COND]
{
	\vdash e_0 : \mathbf{bool} \\ \vdash e_1 : \sigma \\ \vdash e_2 : \sigma
}
{
	\vdash \mathbf{if}~e_0~\mathbf{then}~e_1~\mathbf{else}~e_2 : \sigma 
}
\end{mathpar} 
To show how this works, let us prove that the type of $\mathbf{if}~\mathit{true}~\mathbf{then}~(23 + 42)~\mathbf{else}~4$ is $\mathbf{nat}$. If we think of the expression as a tree, we have a conditional expression at the top. This means that we need to use the \textsf{T-COND} rule first:
\begin{mathpar}
\inferrule*[Right=T-COND]
{
	\vdash \mathit{true} : \mathbf{bool} \\ \vdash (23 + 42) : \mathbf{nat} \\ \vdash 4 : \mathbf{nat}
}
{
	\vdash \mathbf{if}~\mathit{true}~\mathbf{then}~(23 + 42)~\mathbf{else}~4 : \mathbf{nat}
}
\end{mathpar}
Here we have replaced the meta variables $e_0$, $e_1$, and $e_2$ with the corresponding expressions. Next we need to show that all of the three conditions hold. We work our way from left to right. $\vdash \mathit{true} : \mathbf{bool}$ holds according to the \textsf{T-BOOL} rule:
\begin{mathpar}
\inferrule*[Right=T-COND]
{
	\inferrule*[Right=T-BOOL] { }{\vdash \mathit{true} : \mathbf{bool}} \qquad \\ \vdash (23 + 42) : \mathbf{nat} \\ \vdash 4 : \mathbf{nat}
}
{
	\vdash \mathbf{if}~\mathit{true}~\mathbf{then}~(23 + 42)~\mathbf{else}~4 : \mathbf{nat}
}
\end{mathpar}
We use \textsf{T-ADD} to show that $\vdash (23+42) : \mathbf{nat}$:
\begin{mathpar}
\inferrule*[Right=T-COND]
{
	\inferrule*[Right=T-BOOL] { }{\vdash \mathit{true} : \mathbf{bool}} \qquad \\
	\inferrule*[Right=T-ADD] { \vdash 23 : \mathbf{nat} \\ \vdash 42 : \mathbf{nat} } { \vdash (23 + 42) : \mathbf{nat} } \qquad \\ 
	\vdash 4 : \mathbf{nat}
}
{
	\vdash \mathbf{if}~\mathit{true}~\mathbf{then}~(23 + 42)~\mathbf{else}~4 : \mathbf{nat}
}
\end{mathpar}
\textsf{T-ADD} has two conditions, both of which we can prove using \textsf{T-VAL}:
\begin{mathpar}
\inferrule*[Right=T-COND]
{
	\inferrule*[Right=T-BOOL] { }{\vdash \mathit{true} : \mathbf{bool}} \qquad \\
	\inferrule*[Right=T-ADD] { \inferrule*[Right=T-VAL] { } {\vdash 23 : \mathbf{nat}} \qquad \\ \inferrule*[Right=T-VAL] { } {\vdash 42 : \mathbf{nat}} } { \vdash (23 + 42) : \mathbf{nat} } \qquad \\ 
	\vdash 4 : \mathbf{nat}
}
{
	\vdash \mathbf{if}~\mathit{true}~\mathbf{then}~(23 + 42)~\mathbf{else}~4 : \mathbf{nat}
}
\end{mathpar}
Lastly, we use \textsf{T-ADD} one more time to prove that $\vdash 4 : \mathbf{nat}$:
\begin{mathpar}
\inferrule*[Right=T-COND]
{
	\inferrule*[Right=T-BOOL] { }{\vdash \mathit{true} : \mathbf{bool}} \qquad \\
	\inferrule*[Right=T-ADD] { \inferrule*[Right=T-VAL] { } {\vdash 23 : \mathbf{nat}} \qquad \\ \inferrule*[Right=T-VAL] { } {\vdash 42 : \mathbf{nat}} } { \vdash (23 + 42) : \mathbf{nat} } \qquad \\ 
	\inferrule*[Right=T-VAL] { } {\vdash 4 : \mathbf{nat}}
}
{
	\vdash \mathbf{if}~\mathit{true}~\mathbf{then}~(23 + 42)~\mathbf{else}~4 : \mathbf{nat}
}
\end{mathpar}

The type inference problem can now be formalised as: given an arbitrary expression $e$, can we assign a type $\tau$ to it such that the typing relation holds:
\begin{displaymath}
\vdash e : \tau
\end{displaymath}

\begin{parts}
\part In a language of your choice, implement suitable data structures to represent expressions and types in our language. You should also implement the evaluation function.
\part Turn the deduction system described above into an algorithm and implement it in a language of your choice.
\part What is the time complexity of your type inference algorithm?
\part \emph{Type checking} is a similar problem to type inference. It asks: given an expression $e$ and a type $\tau$, is $\tau$ a suitable type for $e$? Show how type checking can be implemented by reducing it to type inference.
\part Suppose that we want to lift the restriction on conditionals and allow each branch to return a potentially different type. Discuss how this could be implemented in the type system. Describe all changes you would make to the type syntax and the typing rules.
\end{parts}

\section{Lazy evaluation}

\question Languages such as Java and ML are \emph{strict}. When we call a function, its arguments will be evaluated first before they are passed on to the function. Suppose we have a function in ML named $\mathit{foo}$:
\begin{displaymath}
\mathbf{fun}~\mathit{foo}(x,y) = x + y
\end{displaymath}
If we call this function as $\mathit{foo}(4+8,16-15)$, evaluation will take place in the following order:
\begin{displaymath}
\begin{array}{cl}
 & \mathit{foo}(4+8,16-15) \\
\Rightarrow & \mathit{foo}(12,16-15) \\
\Rightarrow & \mathit{foo}(12,1) \\
\Rightarrow & 12 + 1
\end{array}
\end{displaymath}
In a \emph{lazy} language such as Haskell, we do not evaluate anything until it is needed:
\begin{displaymath}
\begin{array}{cl}
 & \mathit{foo}(4+8,16-15) \\
\Rightarrow & (4+8) + (16-15)
\end{array}
\end{displaymath}