\documentclass[10pt,a4paper,fleqn]{exam}
\usepackage[latin1]{inputenc}
\usepackage{amsmath}
\usepackage{amsfonts}
\usepackage{amssymb}
\usepackage{graphicx}
\usepackage{titlesec}
\usepackage{hyperref}
\usepackage{fancyeq}
\usepackage{tikz}
%\usepackage{tikz-uml}
\usepackage{mathpartir}

\usepackage[sc]{mathpazo}
\linespread{1.05}         % Palatino needs more leading (space between lines)
\usepackage[T1]{fontenc}

% some format settings
% for hard-bound final submission, use:
%\setlength{\oddsidemargin}{4.6mm}     % 30 mm left margin - 1 in
% for soft-bound version and techreport, use instead:
\setlength{\oddsidemargin}{-0.4mm}    % 25 mm left margin - 1 in
\setlength{\evensidemargin}{\oddsidemargin}
\setlength{\topmargin}{-5.4mm}        % 20 mm top margin - 1 in
\setlength{\textwidth}{160mm}         % 20/25 mm right margin
\setlength{\textheight}{237mm}        % 20 mm bottom margin
\setlength{\headheight}{5mm}
\setlength{\headsep}{5mm}
\setlength{\parindent}{0mm}
\setlength{\parskip}{\medskipamount}
\renewcommand\baselinestretch{1.2} % thesis format (not needed for techreport)
% don't let large figures hijack entire pages
\renewcommand\topfraction{.9}
\renewcommand\textfraction{.1}
\renewcommand\floatpagefraction{.8}

\pagestyle{headandfoot}
%\pointsinrightmargin
%\pointname{ marks}
%\marginpointname{ marks}

\marksnotpoints 

\hypersetup{  
  urlcolor=black,
  linkcolor=black,
  colorlinks=true  
}

\titlelabel{\llap{\thetitle\quad}}

\newcommand {\lbrac} {\makebox[0pt]{[\kern-1ex[}}
\newcommand {\rbrac} {\makebox[0pt]{]\kern-1ex]}}
\newcommand{\denote}[1]{\lbrac~#1~\rbrac}

\begin{document}

\begin{center}
\Large Algorithms \\
\LARGE \textbf{Exercise 4 (Bonus): Data structures II} \\
\end{center}

\hrule

\vspace{0.5cm}

\marksnotpoints
\pointsdroppedatright
\marksnotpoints
\marginpointname{ \points}

%Some ideas: file systems, scene graphs? operating system stuff (mixing the tress and queues, heaps, etc.?)

%\section{Quadtrees}

%\question A \emph{quadtree} 

%Data structures are often explained in an imperative setting. In this exercise, we will look at binomial heaps and red-black trees in a purely functional setting\footnote{This exercise sheet is largely based on \emph{Purely Functional Data Structures} by Chris Okasaki.}. For this exercise, we will use a pure subset of ML\footnote{You may use other languages, as long as you impose the same restrictions there.}. In other words, we are not allowed to use references/mutable values.

\begin{questions}

\section{Thanks, James!}

\question Is it true or false that $\displaystyle 2^r = \sum_{n=0}^{r} \binom{n}{r}$? Give a proof by induction if it is true or a counterexample if not.
\question Suppose 

\end{questions}
\end{document}